%
\documentclass{beamer} 
%
\usepackage{beamerthemesplit}
%\usepackage[pdftex]{graphicx}

\definecolor{links}{HTML}{2A1B81}
\hypersetup{colorlinks,linkcolor=,urlcolor=links}
\usetheme{Frankfurt}
%\usecolortheme{beaver}     % color theme
\useoutertheme{infolines}   % defines the head and the footline of each slide
\useinnertheme{rectangles}

\usepackage{listings}

\usepackage{etex, xy}
\xyoption{all} 

\setbeamertemplate{navigation symbols}{}
%\setbeamertemplate{footline}{}


%
%
\title[LRC]
       {Fall 2016 Python Training}
\author[SSAI]{Carlos Cruz, Jules Kouatchou, and Brent Smith\\
        {\tiny carlos.a.cruz@nasa.gov, Jules.Kouatchou@nasa.gov, brent.smith@nasa.gov}}
 \institute[GSFC]{\NASAlogo \\ Goddard Space Flight Center}
\date{September 19, 2016}

\def\NASAlogo{%
  \includegraphics[scale=0.35,height=30pt]{Meatball.png}%
}

\def\logo{%
  \includegraphics[scale=0.07,height=10pt]{Meatball.png}%
}



\begin{document}

%%%%%%%%%%%%%%%%%%%%%%%%%%%%%%%%%%%%%%%%%%%%%%%%%%%%%%%%
%%%% This is used to add line number to source code %%%%
%%%%%%%%%%%%%%%%%%%%%%%%%%%%%%%%%%%%%%%%%%%%%%%%%%%%%%%%
\lstset{
        language=python,        % language of the code
        basicstyle=\small\ttfamily, % size of the fonts used for the code
        %basicstyle=\normalsize, % size of the fonts used for the code
        numbers=left,           % where to put line numbers
        numberstyle=\tiny,     % size of the fonts used for line numbers
        stepnumber=1, 
        numbersep=8pt,
        showstringspaces=false, % underline spaces within strings
        %aboveskip=-60pt,
        numbersep=3pt,          % how far the line-numbers are from the code
        frame=leftline,         % addition of a left frame line on source code
        keywordstyle=\color{violet}\bfseries, % color keywords
        commentstyle=\color{blue}\bfseries    % color comments
        }

%%%%%%%%%%%%%%%%%%%%%%%%%%%%%%%%%%%%%%%%%%%%%%%%%%%%%%%%%%%
%%%% Title page slide

\frame[plain]{\titlepage}

%\frame{\frametitle{Table of contents}\tableofcontents}

%%%%%%%%%%%%%%%%%%%%%%%%%%%%%%%%%%%%%%%%%%%%%%%%%%%%%%%%%%%
\section{General Remarks}

%%%%%%%%%%%%%%%%%%%%%%%%%%%%%%%%%%%%%%%%%%%%%%%%%%%%%%%%%%%

\frame[containsverbatim]{
  \frametitle{Who Are We?}

  \begin{itemize}
  \item Carlos A. Cruz (Computational Scientist, Occasional Python User)
  \item Jules Kouatchou (Computational Scientist, Occasional Python User)
  \item E. Brent Smith (Programmer and Scientist, Daily Python User)
  \end{itemize}


}

%%%%%%%%%%%%%%%%%%%%%%%%%%%%%%%%%%%%%%%%%%%%%%%%%%%%%%%%%%%

\frame[containsverbatim]{
   \frametitle{Training Objectives}

   We want to:
%
   \begin{enumerate}
     \item Introduce the basic concepts of Python programming
     \item Create functions and modules
     \item Manipulate Python objects (list, tuple, arrays, etc.) 
     \item Handle files
     \item Do plotting
  \end{enumerate}
}

%%%%%%%%%%%%%%%%%%%%%%%%%%%%%%%%%%%%%%%%%%%%%%%%%%%%%%%%%%%

\frame[containsverbatim]{
  \frametitle{What we will Cover}

  \begin{enumerate}
  \item Core principles of Python: Day 1
  \item Data manipulation with Python: Day 2 

  \end{enumerate}

}

%%%%%%%%%%%%%%%%%%%%%%%%%%%%%%%%%%%%%%%%%%%%%%%%%%%%%%%%%%%

\frame[containsverbatim]{
  \frametitle{Target Audience}

%  \begin{itemize}
%  \item People with little or no knowlege of Python: Day 1, Day 2 and Day 3
%  \item Intermediate Python users: Day 2 and Day 3
%  \item Advanced Python users: Day 3
%  \end{itemize}

\begin{center}
\begin{tabular}{|l|c|c|}\hline\hline
Python User  & Day 1 & Day 2  \\ \hline\hline
Beginner       &  X      & X      \\
Intermediate &          & X     \\ \hline\hline
\end{tabular}
\end{center}

}





%%%%%%%%%%%%%%%%%%%%%%%%%%%%%%%%%%%%%%%%%%%%%%%%%%%%%%%%%%%


\frame[containsverbatim]{
  \frametitle{What We Expect from You}

  \begin{itemize}
   \item Have your own laptop.
   \item Install on your system a Python distribution (such as Anaconda) that should at least have iPython, Numpy, Matplotlib.
   \item Install the package Git
   \item Be able to create/edit files on your platform
   \item Do the examples yourself as we move along
   \item Ask questions
  \end{itemize}

}

%%%%%%%%%%%%%%%%%%%%%%%%%%%%%%%%%%%%%%%%%%%%%%%%%%%%%%%%%%%
\frame[containsverbatim]{
\frametitle{Obtaining the Materials}
%


You can obtain all the materials by issuing the command:
  \begin{center}
  {\tt git clone https://github.com/JulesKouatchou/LRC\_Fall16}
  \end{center}
  
You will then get on your platform the directory {\tt LRC\_Fall16}.
  %	


}

%%%%%%%%%%%%%%%%%%%%%%%%%%%%%%%%%%%%%%%%%%%%%%%%%%%%%%%%%%%


\frame[containsverbatim]{
  \frametitle{Beyond the Agenda}

There are few topics that will not be covered but are worth looking at.
Presentations were prepared on:

\begin{enumerate}
\item Datetime Module
\item F2Py 
\end{enumerate}

}

%%%%%%%%%%%%%%%%%%%%%%%%%%%%%%%%%%%%%%%%%%%%%%%%%%%%%%%%%%%


\frame[containsverbatim]{
  \frametitle{Informal Self-Assessment}

At the end of Day 1, you might consider taking a 25-question test at:

  \begin{center}
  \url{http://www.afterhoursprogramming.com/tests/practice/Python/} \\
  \href{http://www.afterhoursprogramming.com/tests/practice/Python/}{}
  \end{center}
  

}

%\frame[containsverbatim]{
%  \frametitle{Any Question?}
%
%To contact us, please send an email to:
%\begin{center}
%\href{mailto:pythonbootcamp@bigbang.gsfc.nasa.gov}{pythonbootcamp@bigbang.gsfc.nasa.gov} 
%\end{center}
%
%}



%%%%%%%%%%%%%%%%%%%%%%%%%%%%%%%%%%%%%%%%%%%%%%%%%%%%%%%%%%%

%\frame[containsverbatim]{
%  \frametitle{}
%}

\frame[containsverbatim,allowframebreaks]{
  \frametitle{Useful Pointers}
  \begin{thebibliography}{9} 
   \setbeamertemplate{bibliography item}[online]
   \bibitem{O1} Python Programming - Introduction 
    \newblock  
     \url{http://www.youtube.com/watch?v=72RKMMyLxS8} \\
     \href{http://www.youtube.com/watch?v=72RKMMyLxS8}{}
%
   \bibitem{O2} A Hands-On Introduction to Python for Beginning Programmers 
    \newblock  
     \url{https://www.youtube.com/watch?v=rkx5_MRAV3A} \\
     \href{https://www.youtube.com/watch?v=rkx5_MRAV3A}{}
%
    \bibitem{O3} A Beginner's Python Tutorial  
    \newblock  
     \url{http://www.sthurlow.com/python/} \\
     \href{http://www.sthurlow.com/python/}{} 
%
    \bibitem{O4} Invent with Python   
    \newblock  
     \url{http://inventwithpython.com/chapters/} \\
     \href{http://inventwithpython.com/chapters/}{}  
%
%
    \bibitem{O5} Think Python: How to Think Like a Computer Scientist   
    \newblock  
     \url{http://greenteapress.com/thinkpython/html/index.html} \\
     \href{http://greenteapress.com/thinkpython/html/index.html}{}  
         
%
  \setbeamertemplate{bibliography item}[book]
    \bibitem{Langtangen}
    Hans Petter Langtangen.
    \newblock {\em A Primer on Scientific Programming with Python}.
    \newblock Springer, 2009.
 \setbeamertemplate{bibliography item}[book]
    \bibitem{Lin2012}
    Johnny Wei-Bing Lin.
    \newblock {\em A Hands-On Introduction to Using Python in the Atmospheric and Oceanic Sciences}.
    \newblock http://www.johnny-lin.com/pyintro, 2012.
  \setbeamertemplate{bibliography item}[book]
    \bibitem{McCormack}
     Drew McCormack.
    \newblock {\em Scientific Scripting with Python}.
    \newblock 2009.
  \end{thebibliography}
}

%%%%%%%%%%%%%%%%%%%%%%%%%%%%%%%%%%%%%%%%%%%%%%%%%%%%%%%%%%%



\end{document}
